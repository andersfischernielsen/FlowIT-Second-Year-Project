\subsection{Workflows}
A workflow is a standardized work description of a process that is executed repeatedly. \newline
An instance of a workflow describes an ongoing process with several stages and describes in which order these states must be reached. A workflow contains several tasks that must be executed for the workflow to either finish in a successful or unsuccessful state. The workflow must also guarantee that the process finishes. \newline
Certain types of workflow system notations exist. Today, most workflow systems are based on the flow-oriented process notation\footnote{Concurrency and Asynchrony, \textit{Søren Debois, Thomas Hildebrandt, and Tijs Slaats}, IT University of Copenhagen}. These, unlike those described with DCR graphs, do not support that the design of the workflow can change dynamically. Furthermore the user cannot deviate from the plan even if it made sense to do so business wise. One of the strengths of DCR graph-based workflows is that any route can be taken to complete the workflow as long as the rules of the workflow are followed.