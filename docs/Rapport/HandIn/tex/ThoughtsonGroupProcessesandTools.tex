\subsection{Thoughts on Group Processes and Tools}
The team strived to run the project as a SCRUM project. This meant that the team held sprint planning meetings, did task division, used a SCRUM board for each sprint, had daily standup meetings, and made sprint retrospectives. 

The team has primarily worked like this. The benefits of using SCRUM were that it was clear to see what tasks were currently due, which tasks were in progress, and which members were responsible of finishing what. \\

In the last part of the project the team realized that it would be beneficial to have a common objective to pursue during a sprint. Not just in terms of individual completed tasks, but also in terms of what should be presentable to a fictional product owner. 

The lesson was that the team overlooked the extensive task of joining finished work and making sure that the individual modules cooperated as intended. 

If the team had a product owner - with whom weekly meetings could have been held - the team might have been forced to focus more on deliverables instead of simply completing individual modules. \\

GitHub was used for version control, and branches were used to implement new features. The branch would then be merged into the master branch when it was deemed finished. 

This enabled easier reversion and made merging bigger changes easier. Isolating a feature on a single branch also helped the members get an overview of the state of the system during implementation. 