\subsection{Scope}
The system should fulfill the following requirements:
\begin{itemize}
\item The system must be able to handle generic DCR graphs and the logic they follow. This is to be tested on a DCR graph produced from a workflow of a Brazilian hospital and a DCR graph given by our lecturers.
\item The system must be able to create, reconfigure, and delete workflows.
\item The architecture of the system must resemble a peer-to-peer distribution.
\item The implementation should be based on REST services. 
\item The system should provide a graphical user interface (UI) to the user. 
\item The UI should be able to present an event log or history, that describes which operations have been executed, aborted or changed in a given workflow. Because “log” and “lock” sounds the same the team chose to call the “logging” feature “History”.
\item The system must persist data such that it can be restarted and handle isolated occurrences of crashes.
\item The system must implement concurrency control that ensures that an illegal state cannot be reached.
\item The system must implement role based access control (RBAC) that ensures that a user is only able to make changes that the given user has permission to make. 
\item The system must allow reconfiguration of already existing events and their relations. This can be achieved by deleting and then recreating the individual events. 
\end{itemize}