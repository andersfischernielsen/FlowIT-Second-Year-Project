\subsection{Design of Client}
This section presents the design of the Client. When the team designed the system, the intent was to make the Client "as stupid as possible".

The intention was that the Client should provide a visual representation of raw data with a thin shell of error handling. To some extent this design was achieved.

The only data logic the Client has implemented is the filtering of workflows and events.
The rest of the Client is just lists of data, which is presented in a design which the team believes presents a good overview. 

\subsubsection{MVVM-principles in Client}
The Client was implemented with the Model-View-ViewModel (MVVM) design principle in mind. Using Windows Presentation Foundation (WPF) this is made possible by “binding” views to the properties of view models.

The MVVM principle makes it possible to separate the views from both the model and the business logic. Additionally, it converts data into a form suitable for presentation in the view, by adding a layer - the view model - in between the view and the model. The view model also maps user actions to logic functionality.

In the system, all view models inherits from a base class, which enables invoking a method whenever a property changes. WPF uses this to update the views in the GUI, whenever data changes.

Using MVVM also enables easier testing, since all user interface actions are methods of the view models.