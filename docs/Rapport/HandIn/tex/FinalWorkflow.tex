\subsection{Final Workflow \label{sec:FinalWorkflow}}

\begin{figure}
\centering
\includegraphics[width=0.5\linewidth]{Figures/strawberry}
\caption{\label{fig:FinalWorkflow} Final workflow}
\end{figure}

The final workflow turned out more complex than the initial interpretation. After feedback from the external partner, the team has added the role \textit{Nurse}, which was not described in the textual description. Furthermore we divided the examinations and the appointments to different events. Although similar, they are not identical. 
\begin{itemize}
\item All events, except \textbf{DownloadScheduling}, \textbf{ReportToUBS}, and \textbf{CheckInPatient} are excluded to begin with.
\item All events, except \textbf{DownloadScheduling} and \textbf{ReportToUBS}, are self excluding, meaning once they are carried out, they cannot be executed again, unless they are subsequently included by some other activity. 
\item The only requirement is, once a patient has been checked in, the government will \textit{eventually} be charged, this is clarified by the response relation between the events \textbf{CheckInPatient} and \textbf{ChargeGovernment}. This may happen either after the patient has gone through the workflow a number of times or a single time. Notice that we are able to leave out a lot of response relations, because the single response relation drives the workflow towards \textbf{ChargeGovernment}.
\end{itemize}

In the following section, the final workflow and its different branches are described. Before branching out, every patient entering the hospital is handled the same way by the hospital employees.

\begin{enumerate}
\item \textbf{CheckInPatient}: Requires that the government will eventually be charged, hence the response relation to \textbf{ChargeGovernment}. It excludes \textbf{ReferToROTA}; why will be explained later. 
\item \textbf{ReceiveAttendanceRecord:} \textbf{ReceiveAttendanceRecord} has inclusion relations to both \textbf{PutPatientInAppointmentQueue} and \textbf{PutPatientInExaminationQueue}. 
\item \textbf{PutPatientInExaminationQueue} and \textbf{PutPatientInAppointmentQueue} exclude each other, since the given hospital description allows only one of them to occur. \newline
As soon as one of these event are executed, one of the two branches are chosen.
\end{enumerate}

From here on, the workflow splits into two separate branches. We first describe the branch related to a patient that is put in an appointment queue: 
\begin{enumerate}
\item \textbf{PutPatientInAppointmentQueue}: This event includes the \textbf{PreparePatientForAppointment} event and excludes itself.
\item \textbf{PreparePatientForAppointment}: This event includes \textbf{PerformAppointment} and excludes itself.
\item \textbf{PerformAppointment}: This event includes \textbf{CreateMedicalReport} and excludes itself.  
\item \textbf{CreateMedicalReport:} From this point forth, there are five different paths a patient could take, and hence \textbf{CreateMedicalReport} has five inclusion relations to the events explained in point 5 to 8 as well as the event \textbf{ChargeGovernment}.
\item \textbf{RecommendAppointmentWithSpecialist}: Because a \textit{Specialist} may either  recommend an appointment with another specialist, request more exams or both, \textbf{RecommendAppointmentWithSpecialist} does not exclude \textbf{RequestMoreExams}. However, it excludes \textbf{RecommendMedicationAndHavePatientReturn}, \textbf{RecommendSurgery}, and \textbf{ChargeGovernment}, as these are not legal options from here on. \textbf{ChargeGovernment} is not an option, since this will not take place at this point. \textbf{RecommendAppointmentWithSpecialist} includes \textbf{ReferToROTA}, since a specialist may refer the patient to ROTA. 
\item \textbf{RequestMoreExams}: Has similar relations as those described in point 5, but starting from \textbf{RequestMoreExams} instead.
\item R\textbf{ecommendMedicationAndHavePatientReturn} and \textbf{RecommendSurgery} both exclude the events mentioned in point 5 and 6, \textbf{ChargeGovernment} as well as each other. They have include relations to both \textbf{ReferToROTA} and \textbf{CheckInPatient}. If the patient is checked in again, we do not want to \textbf{ReferToROTA} to be available for execution anymore, and this explains the exclusion relation from \textbf{CheckInPatient} to \textbf{ReferToROTA} mentioned earlier in the description of the final workflow.
\item \textbf{ReferToROTA} includes only \textbf{CheckInPatient}, as this is what is allowed to happen from here on. Notice, that we do not model whether the referral is approved or not; but we assume that in either case, the patient will return to the hospital. We do not include the rejection or approval of the ROTA referral as this takes place outside of the hospital.
\end{enumerate}

Now, the second branch will be described. To clarify, this is the branch in which a patient was put into an examination queue:
\begin{enumerate}
\item \textbf{PutPatientInExaminationQueue}: This event includes the \textbf{PreparePatientForExamination} event and excludes itself.
\item \textbf{PreparePatientForExamination}: This event includes \textbf{PerformExamination} and excludes itself.
\item \textbf{PerformExamination}: This event includes \textbf{CreateMedicalExaminationReport}, and excludes itself.
\item \textbf{CreateMedicalExaminationReport}: This event includes \textbf{ChargeGovernment}, and excludes itself. \textbf{ChargeGovernment} has its pending value set to true, because of the response relation from \textbf{CheckInPatient}.
\end{enumerate}

\textbf{ChargeGovernment} is the event which puts the workflow in a successful state, since the \textit{Pending} value of \textbf{ChargeGovernment} is set to true, when a patient is checked in. When executing this event the government is finally charged. The government is not charged with each visit to the hospital, but is eventually charged for the full treatment of a patient. \newline
It is possible for both branches to end in this state.