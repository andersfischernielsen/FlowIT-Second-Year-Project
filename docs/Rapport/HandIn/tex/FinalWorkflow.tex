\subsection{Final Workflow}

\begin{figure}
\centering
\includegraphics[width=0,5\linewidth]{Figures/strawberry}
\caption{\label{fig:FinalWorkflow} Final workflow}
\end{figure}

The final workflow turned out more complex than the initial interpretation. After feedback from the external partner, the team has added the role \textit{Nurse}, which was not described in the textual description. Furthermore we divided the examinations and the appointments to different events. Although similar, they are not identical. 
\begin{itemize}
\item All events, except \textbf{DownloadScheduling}, \textbf{ReportToUBS}, and \textbf{CheckInPatient} are excluded to begin with.
\item All events, except \textbf{DownloadScheduling} and \textbf{ReportToUBS}, are self excluding, meaning once they are carried out, they cannot be executed again, unless they are subsequently included by some other activity. 
\item The only requirement is, once a patient has been checked in, the government will \textit{eventually} be charged, this is clarified by the response relation between the events \textbf{CheckInPatient} and \textbf{ChargeGovernment}. This may happen either after the patient has gone through the workflow a number of times or a single time. Notice that we are able to leave out a lot of response relations, because the single response relation drives the workflow towards \textbf{ChargeGovernment}.
\end{itemize}

In the following section, the final workflow and its different branches are described. Before branching out, every patient entering the hospital is handled the same way by the hospital employees.

\begin{enumerate}
\item \textbf{CheckInPatient}: Requires that the government will eventually be charged, hence the response relation to \textbf{ChargeGovernment}. It excludes \textbf{ReferToROTA}; why will be explained later. 
\item \textbf{ReceiveAttendanceRecord:} \textbf{ReceiveAttendanceRecord} has inclusion relations to both \textbf{PutPatientInAppointmentQueue} and \textbf{PutPatientInExaminationQueue}. 
\item \textbf{PutPatientInExaminationQueue} and \textbf{PutPatientInAppointmentQueue} exclude each other, since the given hospital description allows only one of them to occur. \newline
As soon as one of these event are executed, one of the two branches are chosen.
\end{enumerate}

From here on, the workflow splits into two separate branches. We first describe the branch related to a patient that is put in an appointment queue: 
\begin{enumerate}
\item \textbf{PutPatientInAppointmentQueue}: This event includes the \textbf{PreparePatientForAppointment} event and excludes itself.
\item PreparePatientForAppointment: This event includes PerformAppointment and excludes itself.
\end{enumerate}